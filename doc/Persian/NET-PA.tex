\documentclass{article}
    \usepackage{amsmath}
    \usepackage{graphicx}
    \usepackage{HandoutTemplate}
    
    \DefaultMathsDigits
    
    \settextfont[Scale=1.1]{B Nazanin}
    %\settextfont[Scale=1.1]{B Nazanin.ttf}
    \setlatintextfont{Times New Roman}
    \setdigitfont{Times New Roman}
    %\setlatindigitfont{Times New Roman}
    
    
    \DeclareMathOperator{\var}{Var}
    
    \begin{document}
    \handout
    {\lr{CE-40695}}
    {۱}
    {آذر ۱۳۹7}
    {پیاده سازی یک شبکه     \lr{P2P}}
    {}
    
    \vspace{0.3cm}
    
    در این تمرین یک 
    \lr{DNS Server}\footnote{\lr{Domain Name Server}}
    پیاده‌سازی ‌می‌کند.
    این پروتکل روشی توزیع‌شده برای مربوط‌کردن اسم‌ها به آی‌پی‌ها است.
    \\
    برنامه‌ی شما باید روی پورت ۱۵۳۵۳ برای درخواست‌ها گوش‌کند و به هر درخواست پاسخ‌دهد. برای هر درخواست برنامه‌ی شما باید یک فایل متنی هم تولید بکند. جزئیات هر بخش در ادامه توضیح‌داده‌می‌شوند.
    \\
    \section{مقدمه}
    \section{مقدمه}
	\subsection{\lr{Stream}}
	در تابع     \lr{constructor} این کلاس ابتدا با متد \lr{is\_valid} فرمت های \lr{IP} و \lr{Port} را چک میکنیم تا مطمئن شویم به صورت همان فرمت مورد نظر هستند. \lr{server\_in\_buf} همان بافری هست که روی سرور نوشته میشود و هر چند وقت یک بار بایستی چک شود. \lr{Callback Function (cb)} نیز پیام های جدید را به ته \lr{server\_in\_buf} میچسباند \lr{(append)} و در نهایت \lr{Ack} برمیگرداند. این \lr{Ack} باعث میشود هر جا که سوکتی وسط کار قطع شود بفهمد قطع شده است. سپس \lr{tcpserver} را مسازیم, در یک \lr{thread} قرار میدهیم و آن را اجرا میکنیم. \lr{self.nodes} تمام نود هایی هستند که درون ما هستند.\\
\lr{get\_server\_address} آدرس سرور را با آن فرمتی که میخواهیم به ما میدهد.
\\ 
\lr{clear\_in\_buf} بافر سرور را پاک میکند.
\\
\lr{add\_node} نود اضافه میکند.
\\
\lr{remove\_node} نود مشخص شده را از آرایه پاک میکند و سپس  متد \lr{close} نود را اجاره میکند.
\\
\lr{get\_node\_by\_server} آی پی و پورت سرور  یک نود را میگیرد و نود را برمیگرداند. سپس با \lr{parse} کردن آن را به فرمت مد نظر تبدیل میکند.
\\
\lr{add\_message\_to\_out\_buffer} یک آدرس میگیرد و با
\\
\lr{read\_in\_buf} وظیفه دارد \lr{read\_in\_buf} را برگرداند.
\\
\lr{send\_message\_to\_node}
\\
\lr{send\_out\_buf\_message}
\\
	\subsection{\lr{Node}}
    \section{مقدمه}
    برنامه‌ی شما باید درخواست‌های معمولی را که شامل یک آدرس‌ هستند بپذیرد و بدون توجه به بیت 
    \lr{RD} \footnote{\lr{Recursion Desired}}
\lr{ttl}
    ۳۶۰۰
    هستند.
    قسمت
    \lr{RDATA}
    هم همان آی‌پی است.
    \\
    در این نوع درخواست‌ها باید همیشه از آی‌پی برای وصل‌شدن به سرور‌ها استفاده‌کنید و استفاده از آدرس متنی سرور‌ها نمر‌ه‌ای ندارد.
    \\
    نوع دیگری از درخواست‌ها ، درخواست‌های برعکس‌هستند. یعنی در این درخواست‌ها آی‌پی به شما داده‌می‌شود و باید آدرس مربوط به آن آی‌پی را پیدا کنید.
    \\
    نوع معمول این درخواست‌ها به این شکل است که اگر آی‌پی به شکل 
    \lr{a.b.c.d}
    باشد 
    باید به شکل 
    \lr{d.c.b.a.in-addr.arpa}
    در پکت‌باشد و شما هم در هنگام فرستاده کو‌ئری به سرور‌ها بایداز همین فرم‌استفاده‌کنید اما در هنگام بررسی کد شما ‌آی‌پی‌ها به طور معمولی
    (
        همان
        \lr{a.b.c.d}
    )
    به شما داده‌می‌شوند
    و باید آن‌ها را تغییر دهید به سرور بفرستید.
    \\
    فیلد
    \lr{OPCODE}
    باز هم صفر است (هم در پکتی که به کد شما فرستاده‌می‌شود و هم در پکتی که شما برای سرور می فرستید)
    \footnote{ درست‌تر این بود که هر دو یک باشند اما این حالت کار می‌کند و مد نظر است.}
    بقیه فیلد‌ها را مانند قسمت قبل مشخص‌کنید.
    در این قسمت باز هم اگر جوابی از نوع 
    \lr{A}
    بود از آن و اگر چند تا بودند از کم‌ترین مقدار استفاده‌کنید ولی اگر هیچ رکوردی از نوع 
    \lr{A}
    نبود در بین رکورد‌های 
    \lr{NS}
    کم‌ترین را
    (با در نظر گرفتن ترتیب دیکشنری)
    انتخاب‌کنید. البته این رکورد‌ها متنی هستند و آی‌پی نیستند ولی استثناً در این قسمت وصل شدن به اسم مانعی ندارد.
    در نهایت هنگامی که رکوردی از نوع 
    \lr{PTR}
    پیدا شد که فیلد
    \lr{name}
    در آن برابر با آی‌پی آدرس درخواست‌شده (بعد از تبدیل فرمت
    )
    بود آن را به آدرسی که درخواست زده‌بود برگردانید.
    فیلدهای
    \lr{name}
    و
    \lr{rdata}
    در
    پاسخ را برابر با همانی که 
    \lr{DNS Server}
    به شما داد
    ،
    قرار دهید سایر فیلد‌ها را هم مانند بالا تعیین کنید. فقط دقت شود که این دفعه پاسخ از نوع 
    \lr{PTR}
    است.
    \subsection{لاگ}

    برای هر درخواستی که به کد شما زده‌می‌شود باید یک فایل متنی با فرمت 
    \lr{.txt}
    بسازید و اسم آن را هم برابر با 
    \lr{Message ID}
    درخواست قرار‌دهید.
    \\
    فرمت این فایل به شکل زیر است:
    در هر زمانی که به یک آی‌پی وصل می‌شوید باید خط زیر را چاپ کنید:
    \\
    \begin{LTR}
    \begin{verbatim}
    connecting to [IP]
    ===============
    \end{verbatim}
    \end{LTR}	
    بعد از دریافت هر پکت از طرف
    \lr{DNS Server}
    ها
    باید آن پکت را هم در این فایل چاپ کنید که فرمت آن توضیح‌داده‌می‌شود.
    (این کار برای پکت که به عنوان درخواست به کد شما زده‌می‌شود نباید انجام‌شود.)
    هر پکت دارای ۵ بخش است
    \lr{Header}
    ، 
    \lr{Question}
    ، 
    \lr{Answer}
    ، 
    \lr{Authority}
    و 
    \lr{Additional}
    .
    در شروع هرکدام از این قسمت‌ها باید اسم به علاوه‌ی ۱۵ علامت مساوی در فایل بنویسید.
    در پایان هر قسمت هم باید ۱۵ تا مساوی چاپ کنید.
    که به ترتیب در پایین آمده‌اند.
    \begin{LTR}
    \begin{verbatim}
    HEADER
    ===============
    QUESTION
    ===============
    ANSWER
    ===============
    AUTHORITY
    ===============
    \end{verbatim}
    \end{LTR}	
    توجه شود پکت‌های هر قسمت باید بعد از اسم آن قسمت و علامت‌های مساوی نوشته‌شوند.
    فرمت نوشتن پکت‌ها در ادامه توضیح‌داده‌می‌شود.
    برای هر قسمت باید در بین در آکولاد اسم فیلد‌ها و مقدارشان را به ترتیب الفبایی اسم فیلد‌ها و در هر خط یک فیلد، در فایل بنویسید.
    \\
    قسمت
    \lr{Header}
    باید به شکل زیر باشد:
    (مشخصاً مقادیر فرضی هستند و شما باید بر اساس پکت دریافتی آن‌ها را بنویسید.)
    \begin{LTR}
    \begin{verbatim}
    {
    additional count : 14
    answer count : 0
    authority count : 13
    id : 38145
    is authoritative : False
    is response : True
    is truncated : False
    opcode : 0
    question count : 1
    recursion available : False
    recursion desired : False
    reserved : 0
    response code : No Error
    }
    \end{verbatim}
    \end{LTR}	
    اسم فیلد‌ها و فرمت خروجی باید دقیقا مانند بالا باشد. در ادامه سایر قسمت‌ها هم بررسی‌می‌شود و خروجی شما باید از لحاظ ساختار و ترتیب دقیقا مانند آن‌ها باشد.
    \\
    برای فیلد
    \lr{response code}
    مقادیر عددی را با توجه به قالب پایین به متن تبدیل‌کنید.
    \begin{LTR}
    \begin{verbatim}
    0: No Error
    1: Format Error
    2: Server Failure
    3: Name Error
    4: Not Implemented
    5: Refused
    \end{verbatim}
    \end{LTR}	
    برای قسمت 
    \lr{Question}
    به شکل زیر عمل کنید.
    \begin{LTR}
    \begin{verbatim}
    {
    Domain Name : bing.com
    Query Class : 1
    Query Type : 255
    }
    \end{verbatim}
    \end{LTR}
    ساختار قسمت‌های با‌قی‌مانده مانند هم است  پس فقط یکی‌از آن‌ها مثلا 
    \lr{Authority}
    را بررسی‌می‌کنیم.
    در هر کدام از این قسمت‌ها تعدادی پاسخ وجود دارد که هر کدام از آن‌ها می‌توانند از انواع زیر باشند:
    \begin{LTR}
    \begin{verbatim}
    A
    NS
    CNAME
    SOA
    PTR
    MX
    AAAA
    TXT
    \end{verbatim}
    \end{LTR}
    همه‌ی این انواع یک ساختار مشترک به شکل زیر دارند:
    \begin{LTR}
    \begin{verbatim}
    {
    class : ans_class
    name : name
    rdata : rdata
    rdlength : rdlength
    ttl : ttl
    type : ans_type
    }    
    \end{verbatim}
    \end{LTR}
    برای قسمت 
    \lr{type}
    باید اعداد را به اسامی بالا تبدیل‌کنید مگر این‌که عدد جزو تایپ‌های بالا نباشد که در آن صورت عدد را در فایل بنویسید.
    اما قسمت 
    \lr{rdata}
    برای هر کدام فرق می‌کند که در ادامه‌ توضیح‌داده‌می‌شوند.
    \\
    در نوع
    \lr{A}
    این فیلد یک آدرس آی‌پی است که باید به شکل آی‌پی چاپ‌شود.
    \begin{LTR}
    \begin{verbatim}
    rdata : 192.55.83.30
    \end{verbatim}
    \end{LTR}
    در نوع
    \lr{AAAA}
    این فیلد یک آدرس آی‌پی۶ است که باید به شکل آی‌پی۶ چاپ‌شود.
    \begin{LTR}
    \begin{verbatim}
    rdata : 2001:0503:a83e:0000:0000:0000:0002:0030
    \end{verbatim}
    \end{LTR}
    توجه ‌شود جواب را مخفف نکنید و به شکل ۸ عدد ۴ رقمی در مبنای ۱۶ که با 
    «:»
    جدا‌شده‌اند نشان‌دهید
    \\
    نوع‌
    \lr{MX}
    \begin{LTR}
    \begin{verbatim}    
    rdata :
    {
    Mail Exchanger : mta6.am0.yahoodns.net
    Preference : 1
    }
    \end{verbatim}
    \end{LTR}
    نوع
    \lr{SOA}
    \begin{LTR}
    \begin{verbatim}
    rdata :
    {
    Admin MB : hostmaster.yahoo-inc.com
    Expiration Limit : 1814400
    Minimum TTL : 600
    Primary NS : ns1.yahoo.com
    Refresh interval : 3600
    Retry interval : 300
    Serial Number : 2017101218
    }
    \end{verbatim}
    \end{LTR}
    انواع 
    \lr{cname}
    و
    \lr{ptr}
    و
    \lr{ns}
    مانند هم هستند و فقط شامل یک رشته هستند.
    \begin{LTR}
    \begin{verbatim}
    rdata : ns4.yahoo.com
    \end{verbatim}
    \end{LTR}
    نوع 
    \lr{TXT}
    هم شامل یک رشته است اما این رشته به شکل معمولی ذخیره‌شده‌است در سایر انواع این رشته یک 
    \lr{encoding}
    خاص دارد که می‌توانید از لینک‌هایی که در پانوشت صفحه‌ی اول هستند مطالعه‌کنید
    \begin{LTR}
    \begin{verbatim}
    rdata : #v=spf1 redirect=_spf.mail.yahoo.com
    \end{verbatim}
    \end{LTR}
    انواع دیگری از پکت‌های جواب هم موجود هستند که برای آن‌ها لازم نیست
    \lr{rdata}
    را پارس‌کنید و بسیار کم کاربرد هستند و برای آن‌ها کافی است در فایل خروجی 
    \lr{rdata}
    را بنویسید و جلوی آن را خالی بگذارید.
    \begin{LTR}
    \begin{verbatim}
    rdata :
    \end{verbatim}
    \end{LTR}
    و در مقابل فیلد 
    \lr{type}
    هم مقدار عددی نوع آن‌ها را چاپ کنید. به عنوان مثال اگر از نوع 
    \lr{WKS}
    بود به شکل زیر عمل کنید:
    \begin{LTR}
    \begin{verbatim}
    type : 11
    \end{verbatim}
    \end{LTR}
    در نهایت به عنوان مثال فایل اگر ورودی به کد شما آی‌پی‌
    \lr{198.41.0.4}
    بود و پکت درخواستی برای 
    \lr{yahoo.com}
    و 
    با
    آی‌دی
    
    \vfill
    \vspace{1cm}
    $\hfill$ موفق باشید
    \end{document}